\chapter*{Dodatak: Prikaz aktivnosti grupe}
		\addcontentsline{toc}{chapter}{Dodatak: Prikaz aktivnosti grupe}
		
		\section*{Dnevnik sastajanja}
		
		\begin{packed_enum}
			
			\item  sastanak
			\item[] \begin{packed_item}
				\item Datum: 20. listopada 2023.
				\item Prisustvovali: Anamarija Jakoubek, Dino Babić, Jan Grbac, Igor Šoštarko, Karemn Korić, Leonarda Pribanić, Ivan Pavelić
				\item Teme sastanka: Upoznavanje, Osnove projekta
				\begin{packed_item}
					\item Prvo upoznavanje, razmjenjivanje kontaktnih podataka i izrada skupne grupe.
					\item Biranje imena projekta, tehnologije koje će se koristiti, tko bi se kojim dijelom htio baviti te koliko ćemo se često sastajati.
				\end{packed_item}
			\end{packed_item}
			
			\item  sastanak
			\item[] \begin{packed_item}
				\item Datum: 24. listopada 2023.
				\item Prisustvovali: Anamarija Jakoubek, Dino Babić, Jan Grbac, Igor Šoštarko, Karemn Korić, Leonarda Pribanić, Ivan Pavelić
				\item Teme sastanka: Funkcionalni zahtjevi, Dijagrami, Arhitektura i Baza podataka
				\begin{packed_item}
					\item  Navođenje i razrada funkcionalnih i ostalih zahtjeva, koje će mogućnosti imati korisnici, a koji administratori
					\item  Razrađeni dijagrami i napravljene prve skice opisa obrazaca uporabe temeljene na funkcionalnim zahtjevima, također napisani prvi sekvencijski dijagrami koji opisuju kako će aplikacija funkcionirati u danim slučajevima
					\item Dogovor oko arhitekture klijent-poslužitelj koju ćemo koristiti kroz projekt. Za frontend je odabrana React, a za backend kombinacija Java, JavaScript i Spring Boot tehnologija. Prve skice baze podataka, tablica za entitete i postavljanje primarnih i stranih ključeva
				\end{packed_item}
			\end{packed_item}
			
			\item  sastanak
			\item[] \begin{packed_item}
				\item Datum: 7. studenog 2023.
				\item Prisustvovali: Anamarija Jakoubek, Dino Babić, Jan Grbac, Igor Šoštarko, Karemn Korić, Leonarda Pribanić, Ivan Pavelić
				\item Teme sastanka: Dijagrami razreda, Ažuriranje baze podataka
				\begin{packed_item}
					\item Crtanje dijagrama razreda, specifikacija metoda koje povlače Controller razred, definiranje entiteta i servisa koje će entiteti koristiti
					\item Dorada i ispravak baze podataka, dodavanje novih atributa u entitete ili cijele koji zamjenjuju nepotrebne funkcionalnosti baze podataka 
				\end{packed_item}
			\end{packed_item}
			
			\item  sastanak
			\item[] \begin{packed_item}
				\item Datum: 15. studenog 2023.
				\item Prisustvovali: Anamarija Jakoubek, Dino Babić, Jan Grbac, Igor Šoštarko, Karemn Korić, Leonarda Pribanić, Ivan Pavelić
				\item Teme sastanka: Dovršavanje prvog dijela dokumentacije, Objava i testiranje web-aplikacije
				\begin{packed_item}
					\item Ažuriranje slika, dijelova teksta, tablica i ostalih stavki prije prve predaje
					\item Deploy aplikacije na Redner i testiranje funkcionalnosti prijave, registracije, odobravanje registracija i osnovnih funkcionalnosti web-stranice
				\end{packed_item}
			\end{packed_item}
			
			%
			
		\end{packed_enum}
		
		\eject
		\section*{Tablica aktivnosti}

			\begin{longtblr}[
					label=none,
				]{
					vlines,hlines,
					width = \textwidth,
					colspec={X[7, l]X[1, c]X[1, c]X[1, c]X[1, c]X[1, c]X[1, c]X[1, c]}, 
					vline{1} = {1}{text=\clap{}},
					hline{1} = {1}{text=\clap{}},
					rowhead = 1,
				} 
			
				\SetCell[c=1]{c}{} & \SetCell[c=1]{c}{\rotatebox{90}{\textbf{Ivan Pavelić}}} & \SetCell[c=1]{c}{\rotatebox{90}{\textbf{Jan Grbac}}} &	\SetCell[c=1]{c}{\rotatebox{90}{\textbf{Anamarija Jakoubek}}} & \SetCell[c=1]{c}{\rotatebox{90}{\textbf{Karmen Korić}}} &	\SetCell[c=1]{c}{\rotatebox{90}{\textbf{Dino Babić}}} & \SetCell[c=1]{c}{\rotatebox{90}{\textbf{Leonarda Pribanić}}} &	\SetCell[c=1]{c}{\rotatebox{90}{\textbf{Igor Šoštarko}}} \\  
				Upravljanje projektom 		& 5 & 3 &  &  & 4 &  & \\ 
				Opis projektnog zadatka 	&  &  & 2 & 5 &  &  & \\ 
				Funkcionalni zahtjevi       &  &  & 3 & 3 &  &  &  \\ 
				Opis pojedinih obrazaca 	&  &  & 1 & 4 &  &  &  \\ 
				Dijagram obrazaca 			& 4 & 4 & 4 &  & 4 & 4 & 4 \\ 
				Sekvencijski dijagrami 		& 2 & 2 & 2 &  & 2 & 2 & 2 \\ 
				Opis ostalih zahtjeva 		& 1 & 1 & 1 &  & 1 & 1 & 1 \\ 
				Arhitektura i dizajn sustava	&  &  &  &  &  & 3 & 4 \\ 
				Baza podataka				& 2 & 4 & 2 & 2 & 2 & 2 & 2 \\ 
				Dijagram razreda 			&  & 2 &  &  & 4 &  &  \\ 
				Dijagram stanja				&  &  &  &  &  &  &  \\ 
				Dijagram aktivnosti 		&  &  &  &  &  &  &  \\ 
				Dijagram komponenti			&  &  &  &  &  &  &  \\ 
				Korištene tehnologije i alati 		&  &  &  &  &  &  &  \\ 
				Ispitivanje programskog rješenja 	&  &  &  &  &  &  &  \\ 
				Dijagram razmještaja			&  &  &  &  &  &  &  \\ 
				Upute za puštanje u pogon 		&  &  &  &  &  &  &  \\  
				Dnevnik sastajanja 			& 1 &  &  &  &  & 1 &  \\ 
				Zaključak i budući rad 		&  &  &  &  &  &  &  \\  
				Popis literature 			&  &  &  &  &  &  &  \\  
				&  &  &  &  &  &  &  \\ \hline
				\textit{Izrada početne stranice} 	&  &  & 2 & 3 &  &  &  \\
				\textit{Frontend} 					&  &  & 1 & 1 & 5 &  &  \\    
				\textit{Izrada baze podataka} 		& 4 &  &  &  &  &  & 5\\  
				\textit{Spajanje s bazom podataka} 	&  &  &  &  &  & 4 &  \\ 
				\textit{Backend} 					&  & 5 &  &  &  &  &  \\ 
			\end{longtblr}
					
					
		\eject
		\section*{Dijagrami pregleda promjena}
		
		\textbf{\textit{dio 2. revizije}}\\
		
		\textit{Prenijeti dijagram pregleda promjena nad datotekama projekta. Potrebno je na kraju projekta generirane grafove s gitlaba prenijeti u ovo poglavlje dokumentacije. Dijagrami za vlastiti projekt se mogu preuzeti s gitlab.com stranice, u izborniku Repository, pritiskom na stavku Contributors.}
		
	