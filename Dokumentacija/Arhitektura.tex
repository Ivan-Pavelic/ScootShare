\chapter{Arhitektura i dizajn sustava}
		
		\indent Odabrali smo arhitekturu klijent-poslužitelj zbog njezine skalabilnosti i jasne podjele odgovornosti između klijentskog i poslužiteljskog dijela sustava. Ova arhitektura pruža učinkovitu komunikaciju između korisničkog sučelja i servera. \\
		
		\indent Sustav je organiziran kao \textit{\underbar{klijent-poslužitelj}}, pri čemu klijent sadržava korisničko sučelje i logiku za interakciju s korisnikom. S druge strane, poslužitelj obrađuje zahtjeve klijenata, pristupa bazi podataka i upravlja poslovnim logikama. \\
		
		\indent \textit{\underbar{Aplikacija}} je strukturirana na slojevima, gdje frontend koristi React tehnologiju, koja obuhvaća korisničko sučelje i logiku za prikaz podataka korisnicima. Backend, s druge strane, koristi kombinaciju Java, JavaScript i Spring Boot tehnologija, obrađuje poslovnu logiku, upravlja pristupom bazi podataka i komunicira s klijentima. Ova struktura temelji se na \textit{\underbar{Model-View-Controller (MVC)}} arhitekturi, gdje model predstavlja podatke i poslovnu logiku, view je odgovoran za prikazivanje podataka, a controller upravlja korisničkim zahtjevima. \\
		
		\indent \textit{\underbar{Klijent}} radi na različitim uređajima, uključujući računala, pametne telefone i tablete, putem web preglednika. Poslužitelj je smješten u lokalnom data centru s potrebnim resursima za podršku poslužiteljskoj logici i bazi podataka. Baza podataka djeluje kao centralno spremište za pohranu podataka, s pristupom koji je kontroliran od strane poslužitelja. \\
		
		\indent Klijent inicira zahtjeve i prikazuje podatke korisnicima, dok poslužitelj obrađuje zahtjeve, upravlja poslovnim logikama te komunicira s bazom podataka. \\
		\indent Za komunikaciju između klijenta i poslužitelja koristimo HTTP/HTTPS, osiguravajući time sigurnu razmjenu podataka. Za pouzdan prijenos podataka između različitih komponenti sustava koristi se TCP/IP \\
		
		\indent \textit{\underbar{Razvojna okolina}} uključuje korištenje Eclipse i Visual Studio Code za razvoj aplikacije, pružajući programerima učinkovite alate za pisanje, testiranje i održavanje koda tijekom razvojnog ciklusa. \\
				
		\section{Baza podataka}
			
			\noindent Za potrebe našeg sustava koristiti ćemo relacijsku bazu podataka koja svojom strukturom olakšava modeliranje stvarnog svijeta. Gradivna jedinka baze je relacija, odnosno tablica koja je definirana svojim imenom i skupom atributa. Zadaća baze podataka je brza i jednostavna pohrana, izmjena i dohvat podataka za daljnju obradu.
			Baza podataka ove aplikacije sastoji se od sljedećih entiteta:
			
			\begin{packed_item} 
				\item User
				\item Authority
				\item Scooter
				\item Listing
				\item RentalRequest
				\item ImageChangeRequest
				\item Rental
				\item Rating
				\item Transaction
			\end{packed_item}
				
		
			\subsection{Opis tablica}
			

				\textbf{User} entitet sadržava sve važne informacije o korisniku aplikacije. Sadrži atribute: Email, nadimak, lozinku, ime, prezime, broj kartice za plaćanje, presliku osobne iskaznice te potvrdu o nekažnjavanju. Ovaj entitet u vezi je Many-to-One s entitetiom Authority, sa entitetom Scooter, Message, Rental te ImageChangeRequest.
				
				
				\begin{longtblr}[
					label=none,
					entry=none
					]{
						width = \textwidth,
						colspec={|X[10,l]|X[8, l]|X[20, l]|}, 
						rowhead = 1,
					} %definicija širine tablice, širine stupaca, poravnanje i broja redaka naslova tablice
					\hline \SetCell[c=3]{c}{\textbf{User}}	 \\ \hline[3pt]
					\SetCell{LightGreen}userID & INT	&  	Jedinstveni identifikator korisnika  	\\ \hline
					firstName	& VARCHAR &  Ime korisnika 	\\ \hline 
					lastName & VARCHAR &  Prezime korisnika  \\ \hline
					password & VARCHAR &  Hash lozinke  \\ \hline
					CardNumber & VARCHAR &   Broj kreditne kartice   \\ \hline
					email & VARCHAR &   Email adresa korisnika   \\ \hline
					idCard & MEDIUMTEXT &  Kopija osobne iskaznice  \\ \hline
					CertificateOfNo CriminalRecord & MEDIUMTEXT & Potvrda o nekažnjavanju korisnika  \\ \hline 
				\end{longtblr}
				
				\textbf{Authority} Ovaj entitet pohranjuje informacije o ovlastima korisnika. Sadrži atribute: Jedinstveni identifikator autorizacije, identifikator korisnika kojemu ovlast pripada i naziv ovlasti. U vezi je One-To-Many s entitetom User preko atributa jedinstvenog identifikatora korisnika.
				
				\begin{longtblr}[
					label=none,
					entry=none
					]{
						width = \textwidth,
						colspec={|X[10,l]|X[8, l]|X[20, l]|}, 
						rowhead = 1,
					} %definicija širine tablice, širine stupaca, poravnanje i broja redaka naslova tablice
					\hline \SetCell[c=3]{c}{\textbf{Authority}}	 \\ \hline[3pt]
					\SetCell{LightGreen}authorityID & INT	&  	Jedinstveni identifikator autorizacije 	\\ \hline
					\SetCell{LightBlue}userID & INT	&  	Jedinstveni identifikator korisnika  \\ \hline 
					authority	& VARCHAR &  Naziv ovlasti 	\\ \hline 
				\end{longtblr}
				
				\textbf{Scooter} pohranjuje informacije o romobilu koji se iznajmljuje. Sadrži atribute: jedinstvenog identifikatora romobila, jedinstvenog identifikator korisnika,  identifikatori fotografija romobila te fotografije romobila. U vezi je One-to-Many sa entitom User preko njegovog identifikatora te Many-To-One sa entitetom ScooterImages.
				
				\begin{longtblr}[
					label=none,
					entry=none
					]{
						width = \textwidth,
						colspec={|X[10,l]|X[8, l]|X[20, l]|}, 
						rowhead = 1,
					} %definicija širine tablice, širine stupaca, poravnanje i broja redaka naslova tablice
					\hline \SetCell[c=3]{c}{\textbf{Scooter}}	 \\ \hline[3pt]
					\SetCell{LightGreen}scooterID & INT	&  	Jedinstveni identifikator romobila 	\\ \hline
					\SetCell{LightBlue}ownerID & INT	&  	Jedinstveni identifikator korisnika  \\ \hline 
					imageIds	& INT &  Identifikatori fotografija romobila 	\\ \hline 
				\end{longtblr}
				
				\textbf{Listing} Ovaj entitet pohranjuje informacije o oglasu za iznajmljivanje romobila. Sadrži atribute: jedinstveni identifikator oglasa, jedinstveni identifikator romobila, lokaciju romobila, lokaciju povratka romobila, vrijeme povratka romobila, cijenu najma po prijeđenom kilometru, iznos novčane kazne u slučaju ne vraćanja romobila na vrijeme te status oglasa. U vezi je Many-To-One s entitetom Scooter preko identifikatora romobila.
				
				\begin{longtblr}[
					label=none,
					entry=none
					]{
						width = \textwidth,
						colspec={|X[10,l]|X[8, l]|X[20, l]|}, 
						rowhead = 1,
					} %definicija širine tablice, širine stupaca, poravnanje i broja redaka naslova tablice
					\hline \SetCell[c=3]{c}{\textbf{Listing}}	 \\ \hline[3pt]
					\SetCell{LightGreen}listingID & INT	&  	Jedinstveni identifikator oglasa	\\ \hline
					\SetCell{LightBlue}scooterID & INT	&  	Jedinstveni identifikator romobila  \\ \hline 
					location & VARCHAR &  Lokacija romobila	\\ \hline 
					returnLocation & VARCHAR &  Lokacija za povratak romobila	\\ \hline 
					returnByTime & VARCHAR &  Vrijeme do kada romobil mora biti vraćen  \\ \hline
					pricePerKilometer & VARCHAR & Cijena iznajmljivanja po kilometru  \\ \hline
					lateReturnPenalty & VARCHAR & Iznos zakasnine kod vraćanja romobila  \\ \hline
					status & VARCHAR & Status oglasa (aktivno, završeno) \\ \hline
				\end{longtblr}
				
				\textbf{Message} Ovaj entitet sadržava sve važne informacije o porukama između iznajmljivača i unajmitelja. Sadrži atribute: Jedinstveni identifikator poruka, identifikator pošiljatelja i primatelja poruke, vrijeme slanja poruke, tekst poruke i  jedinstveni identifikator oglasa na kojeg se korisnik javlja. Ovaj entitet u vezi je Many-to-One s User preko jedinstvenog identifikatora pošiljatelja, u vezi Many-to-One s User preko jedinstvenog identifikatora primatelja i u vezi Many-to-One s Listing preko jedinstvenog identifikatora oglasa.
				
				\begin{longtblr}[
					label=none,
					entry=none
					]{
						width = \textwidth,
						colspec={|X[10,l]|X[8, l]|X[20, l]|}, 
						rowhead = 1,
					} %definicija širine tablice, širine stupaca, poravnanje i broja redaka naslova tablice
					\hline \SetCell[c=3]{c}{\textbf{Message}}	 \\ \hline[3pt]
					\SetCell{LightGreen}messageID & INT	&  	Jedinstveni identifikator poruke\\ \hline
					\SetCell{LightBlue}userFromID & INT	&  	Jedinstveni identifikator pošiljatelja  \\ \hline
					\SetCell{LightBlue}userToID & INT	&  	Jedinstveni identifikator primatelja  \\ \hline 
					\SetCell{LightBlue}listingID & INT	&  	Jedinstveni identifikator oglasa  \\ \hline
					message & VARCHAR & Poruka korisniku o dostupnosti romobila  \\ \hline
					sentAt & VARCHAR & vrijeme slanja poruke  \\ \hline
				\end{longtblr}
				
				\textbf{ImageChangeRequest} Ovaj entitet pohranjuje zahtjeve korisnika za promjenu slike romobila kojeg su unajmili. Sadrži atribute: Jedinstveni identifikator najma romobila, jedinstveni identifikator slike, opis zamjene slike. U vezi je One-To-One sa ScooterImages te Many-To-One sa Rental entitetom.
				
				\begin{longtblr}[
					label=none,
					entry=none
					]{
						width = \textwidth,
						colspec={|X[10,l]|X[8, l]|X[20, l]|}, 
						rowhead = 1,
					} %definicija širine tablice, širine stupaca, poravnanje i broja redaka naslova tablice
					\hline \SetCell[c=3]{c}{\textbf{ImageChangeRequest}}	 \\ \hline[3pt]
					\SetCell{LightGreen}imageChange RequestID & INT	&  	Jedinstveni identifikator zahtjeva za promjenu slike romobila	\\ \hline
					replacementImage & MEDIUMTEXT & Slika stanja romobila kojom želimo zamijeniti postojeću  \\ \hline
					\SetCell{LightBlue}rentalID & INT	&  	Jedinstveni identifikator najma romobila \\ \hline
					\SetCell{LightBlue}imageID & INT	&  	Jedinstveni identifikator fotografije romobila  \\ \hline
				\end{longtblr}
				
				\textbf{Rental} sadržava sve važne informacije o najmu romobila. Sadrži atribute: jedinstveni identifikator najma romobila, jedinstveni identifikator korisnika, jedinstveni identifikator oglasa, početno vrijeme najma romobila i završno vrijeme najma romobila. Ovaj entitet u vezi je Many-to-One s entitetom User preko identifikatora korisnika, također Many-to-One s entitetom Listing. U vezi One-to-One s entitetom Transaction te u vezi One-to-One s entitetom Rating preko atributa jedinstveni identifikator najma romobila.
				
				\begin{longtblr}[
					label=none,
					entry=none
					]{
						width = \textwidth,
						colspec={|X[10,l]|X[8, l]|X[20, l]|}, 
						rowhead = 1,
					} %definicija širine tablice, širine stupaca, poravnanje i broja redaka naslova tablice
					\hline \SetCell[c=3]{c}{\textbf{Rental}}	 \\ \hline[3pt]
					\SetCell{LightGreen}rentalID & INT	&  	Jedinstveni identifikator najma romobila	\\ \hline
					\SetCell{LightBlue}userID & INT	&  	Jedinstveni identifikator korisnika  \\ \hline
					\SetCell{LightBlue}listingID & INT	&  	Jedinstveni identifikator oglasa  \\ \hline
					rentalTimeStart & VARCHAR & Početno vrijeme najma romobila  \\ \hline
					rentalTimeEnd & VARCHAR & Završno vrijeme najma romobila  \\ \hline
				\end{longtblr}
				
				\textbf{Rating} entitet sadrži sve bitne informacije o ukupnoj ocjeni romobila. Sadrži atribute: Identifikator ocjene, identifikator najma romobila, ocjenu romobila, komentar nakon vožnje te vrijeme postavljanja ocjene. U vezi je One-To-One sa entitetom Rental preko identifikatora najma romobila.
				
				\begin{longtblr}[
					label=none,
					entry=none
					]{
						width = \textwidth,
						colspec={|X[10,l]|X[8, l]|X[20, l]|}, 
						rowhead = 1,
					} %definicija širine tablice, širine stupaca, poravnanje i broja redaka naslova tablice
					\hline \SetCell[c=3]{c}{\textbf{Rating}}	 \\ \hline[3pt]
					\SetCell{LightGreen}ratingID & INT	&  	Jedinstveni identifikator ocjene	\\ \hline
					\SetCell{LightBlue}rentalID & INT	&  	Jedinstveni identifikator najma romobila	\\ \hline
					grade & VARCHAR & Ocjena vožnje romobilom  \\ \hline
					comment & VARCHAR & Komentar nakon vožnje romobilom  \\ \hline
					ratingTime & VARCHAR & Vrijeme ocjenjivanja vožnje \\ \hline
				\end{longtblr}
				
				\textbf{Transaction} Ovaj entitet pohranjuje informacije o jednoj transakciji. Sadrži atribute: identifikator transakcije, identifikator najma, broj prijeđenih kilometara, ukupnu cijenu najma. U vezi je One-To-One s entitetom Rental.
				
				\begin{longtblr}[
					label=none,
					entry=none
					]{
						width = \textwidth,
						colspec={|X[10,l]|X[8, l]|X[20, l]|}, 
						rowhead = 1,
					} %definicija širine tablice, širine stupaca, poravnanje i broja redaka naslova tablice
					\hline \SetCell[c=3]{c}{\textbf{Transaction}}	 \\ \hline[3pt]
					\SetCell{LightGreen}transactionID & INT	&  	Jedinstveni identifikator transakcije	\\ \hline
					\SetCell{LightBlue}rentalID & INT	&  	Jedinstveni identifikator najma romobila	\\ \hline
					totalPrice & VARCHAR & Ukupna cijena najma romobila  \\ \hline
					kilometersPassed & VARCHAR & Kilometri prijeđeni romobilom  \\ \hline
					timeOfTransaction & VARCHAR &  Vrijeme transakcije  \\ \hline
				\end{longtblr}
				
				\textbf{ScooterImages} entitet sadrži fotografije vezane za romobil koji se iznajmljuje. Sadrži atribute: identifikator fotografije, identifikator romobila te samu sliku romobila. U vezi je Many-To-One sa entitetom Scooter preko identifikatora romobila.
				
				\begin{longtblr}[
					label=none,
					entry=none
					]{
						width = \textwidth,
						colspec={|X[10,l]|X[8, l]|X[20, l]|}, 
						rowhead = 1,
					} %definicija širine tablice, širine stupaca, poravnanje i broja redaka naslova tablice
					\hline \SetCell[c=3]{c}{\textbf{ScooterImages}}	 \\ \hline[3pt]
					\SetCell{LightGreen}imageID & INT	&  	Jedinstveni identifikator fotgrafije romobila	\\ \hline
					\SetCell{LightBlue}scooterID & INT	&  	Jedinstveni identifikator romobila  \\ \hline
					image & MEDIUMTEXT & Fotografija romobila  \\ \hline
				\end{longtblr}
				
			
			\subsection{Dijagram baze podataka}
			
				\begin{figure}[H]
					\centering
					\includegraphics[width=0.8\textwidth]{slike/relacijski_model_baza.png}
					\caption{E-R dijagram baze podataka}
					\label{fig:your_label}
				\end{figure}
			
			\eject
			
			
		\section{Dijagram razreda}
		
			\textit{Potrebno je priložiti dijagram razreda s pripadajućim opisom. Zbog preglednosti je moguće dijagram razlomiti na više njih, ali moraju biti grupirani prema sličnim razinama apstrakcije i srodnim funkcionalnostima.}\\
			
			\textbf{\textit{dio 1. revizije}}\\
			
			\textit{Prilikom prve predaje projekta, potrebno je priložiti potpuno razrađen dijagram razreda vezan uz \textbf{generičku funkcionalnost} sustava. Ostale funkcionalnosti trebaju biti idejno razrađene u dijagramu sa sljedećim komponentama: nazivi razreda, nazivi metoda i vrste pristupa metodama (npr. javni, zaštićeni), nazivi atributa razreda, veze i odnosi između razreda.}\\
			
			\textbf{\textit{dio 2. revizije}}\\			
			
			\textit{Prilikom druge predaje projekta dijagram razreda i opisi moraju odgovarati stvarnom stanju implementacije}
			
			
			
			\eject
		
		\section{Dijagram stanja}
			
			
			\textbf{\textit{dio 2. revizije}}\\
			
			\textit{Potrebno je priložiti dijagram stanja i opisati ga. Dovoljan je jedan dijagram stanja koji prikazuje \textbf{značajan dio funkcionalnosti} sustava. Na primjer, stanja korisničkog sučelja i tijek korištenja neke ključne funkcionalnosti jesu značajan dio sustava, a registracija i prijava nisu. }
			
			
			\eject 
		
		\section{Dijagram aktivnosti}
			
			\textbf{\textit{dio 2. revizije}}\\
			
			 \textit{Potrebno je priložiti dijagram aktivnosti s pripadajućim opisom. Dijagram aktivnosti treba prikazivati značajan dio sustava.}
			
			\eject
		\section{Dijagram komponenti}
		
			\textbf{\textit{dio 2. revizije}}\\
		
			 \textit{Potrebno je priložiti dijagram komponenti s pripadajućim opisom. Dijagram komponenti treba prikazivati strukturu cijele aplikacije.}