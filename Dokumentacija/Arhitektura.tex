\chapter{Arhitektura i dizajn sustava}
		
		\textbf{\textit{dio 1. revizije}}\\

		\textit{ Potrebno je opisati stil arhitekture te identificirati: podsustave, preslikavanje na radnu platformu, spremišta podataka, mrežne protokole, globalni upravljački tok i sklopovsko-programske zahtjeve. Po točkama razraditi i popratiti odgovarajućim skicama:}
	\begin{itemize}
		\item 	\textit{izbor arhitekture temeljem principa oblikovanja pokazanih na predavanjima (objasniti zašto ste baš odabrali takvu arhitekturu)}
		\item 	\textit{organizaciju sustava s najviše razine apstrakcije (npr. klijent-poslužitelj, baza podataka, datotečni sustav, grafičko sučelje)}
		\item 	\textit{organizaciju aplikacije (npr. slojevi frontend i backend, MVC arhitektura) }		
	\end{itemize}

	
		

		

				
		\section{Baza podataka}
			
			\noindent Za potrebe našeg sustava koristiti ćemo relacijsku bazu podataka koja svojom strukturom olakšava modeliranje stvarnog svijeta. Gradivna jedinka baze je relacija, odnosno tablica koja je definirana svojim imenom i skupom atributa. Zadaća baze podataka je brza i jednostavna pohrana, izmjena i dohvat podataka za daljnju obradu.
			Baza podataka ove aplikacije sastoji se od sljedećih entiteta:
			
			\begin{packed_item} 
				\item User
				\item Authority
				\item Scooter
				\item Listing
				\item RentalRequest
				\item ImageChangeRequest
				\item Rental
				\item Rating
				\item Transaction
			\end{packed_item}
				
		
			\subsection{Opis tablica}
			

				\textit{Svaku tablicu je potrebno opisati po zadanom predlošku. Lijevo se nalazi točno ime varijable u bazi podataka, u sredini se nalazi tip podataka, a desno se nalazi opis varijable. Svjetlozelenom bojom označite primarni ključ. Svjetlo plavom označite strani ključ}
				
				
				\begin{longtblr}[
					label=none,
					entry=none
					]{
						width = \textwidth,
						colspec={|X[10,l]|X[8, l]|X[20, l]|}, 
						rowhead = 1,
					} %definicija širine tablice, širine stupaca, poravnanje i broja redaka naslova tablice
					\hline \SetCell[c=3]{c}{\textbf{User}}	 \\ \hline[3pt]
					\SetCell{LightGreen}userID & INT	&  	Jedinstveni identifikator korisnika  	\\ \hline
					firstName	& VARCHAR &  Ime korisnika 	\\ \hline 
					lastName & VARCHAR &  Prezime korisnika  \\ \hline 
					Nickname & VARCHAR	&  Nadimak korisnika  \\ \hline
					password & VARCHAR &  Hash lozinke  \\ \hline
					CardNumber & VARCHAR &   Broj kreditne kartice   \\ \hline
					email & VARCHAR &   Email adresa korisnika   \\ \hline
					idCard & MEDIUMTEXT &  Kopija osobne iskaznice  \\ \hline
					CertificateOfNo CriminalRecord & MEDIUMTEXT & Potvrda o nekažnjavanju korisnika  \\ \hline
				\end{longtblr}
				
				\begin{longtblr}[
					label=none,
					entry=none
					]{
						width = \textwidth,
						colspec={|X[10,l]|X[8, l]|X[20, l]|}, 
						rowhead = 1,
					} %definicija širine tablice, širine stupaca, poravnanje i broja redaka naslova tablice
					\hline \SetCell[c=3]{c}{\textbf{Authority}}	 \\ \hline[3pt]
					\SetCell{LightGreen}authorityID & INT	&  	Jedinstveni identifikator autorizacije 	\\ \hline
					\SetCell{LightBlue}UserID & INT	&  	Jedinstveni identifikator korisnika  \\ \hline 
					authority	& VARCHAR &  Autorizacija 	\\ \hline 
				\end{longtblr}
				
				\begin{longtblr}[
					label=none,
					entry=none
					]{
						width = \textwidth,
						colspec={|X[10,l]|X[8, l]|X[20, l]|}, 
						rowhead = 1,
					} %definicija širine tablice, širine stupaca, poravnanje i broja redaka naslova tablice
					\hline \SetCell[c=3]{c}{\textbf{Scooter}}	 \\ \hline[3pt]
					\SetCell{LightGreen}scooterID & INT	&  	Jedinstveni identifikator romobila 	\\ \hline
					\SetCell{LightBlue}ownerID & INT	&  	Jedinstveni identifikator korisnika  \\ \hline 
					imageIds	& INT &  Identifikatori fotografija romobila 	\\ \hline 
					images	& MEDIUMTEXT &  Fotografije romobila 	\\ \hline 
				\end{longtblr}
				
				\begin{longtblr}[
					label=none,
					entry=none
					]{
						width = \textwidth,
						colspec={|X[10,l]|X[8, l]|X[20, l]|}, 
						rowhead = 1,
					} %definicija širine tablice, širine stupaca, poravnanje i broja redaka naslova tablice
					\hline \SetCell[c=3]{c}{\textbf{Listing}}	 \\ \hline[3pt]
					\SetCell{LightGreen}listingID & INT	&  	Jedinstveni identifikator oglasa	\\ \hline
					\SetCell{LightBlue}scooterID & INT	&  	Jedinstveni identifikator romobila  \\ \hline 
					location & VARCHAR &  Lokacija romobila	\\ \hline 
					returnLocation & VARCHAR &  Lokacija za povratak romobila	\\ \hline 
					returnByTime & VARCHAR &  Vrijeme do kada romobil mora biti vraćen  \\ \hline
					pricePerKilometer & VARCHAR & Cijena iznajmljivanja po kilometru  \\ \hline
					lateReturnPenalty & VARCHAR & Iznos zakasnine kod vraćanja romobila  \\ \hline
					status & VARCHAR & Status oglasa (aktivno, završeno) \\ \hline
				\end{longtblr}
				
				\begin{longtblr}[
					label=none,
					entry=none
					]{
						width = \textwidth,
						colspec={|X[10,l]|X[8, l]|X[20, l]|}, 
						rowhead = 1,
					} %definicija širine tablice, širine stupaca, poravnanje i broja redaka naslova tablice
					\hline \SetCell[c=3]{c}{\textbf{RentalRequest}}	 \\ \hline[3pt]
					\SetCell{LightGreen}rentalRequestID & INT	&  	Jedinstveni identifikator zahtjeva	\\ \hline
					\SetCell{LightBlue}userID & INT	&  	Jedinstveni identifikator korisnika  \\ \hline
					\SetCell{LightBlue}listingID & INT	&  	Jedinstveni identifikator korisnika  \\ \hline
					message & VARCHAR & Poruka korisniku o dostupnosti romobila  \\ \hline
					sentAt & VARCHAR & Primatelj kome je poruka poslana  \\ \hline
				\end{longtblr}
				
				\begin{longtblr}[
					label=none,
					entry=none
					]{
						width = \textwidth,
						colspec={|X[10,l]|X[8, l]|X[20, l]|}, 
						rowhead = 1,
					} %definicija širine tablice, širine stupaca, poravnanje i broja redaka naslova tablice
					\hline \SetCell[c=3]{c}{\textbf{ImageChangeRequest}}	 \\ \hline[3pt]
					\SetCell{LightGreen}imageChange RequestID & INT	&  	Jedinstveni identifikator zahtjeva za promjenu slike romobila	\\ \hline
					replacementImage & MEDIUMTEXT & Slika stanja romobila kojom želimo zamijeniti postojeću  \\ \hline
					\SetCell{LightBlue}userID & INT	&  	Jedinstveni identifikator korisnika  \\ \hline
					\SetCell{LightBlue}scooterID & INT	&  	Jedinstveni identifikator romobila  \\ \hline
				\end{longtblr}
				
				\begin{longtblr}[
					label=none,
					entry=none
					]{
						width = \textwidth,
						colspec={|X[10,l]|X[8, l]|X[20, l]|}, 
						rowhead = 1,
					} %definicija širine tablice, širine stupaca, poravnanje i broja redaka naslova tablice
					\hline \SetCell[c=3]{c}{\textbf{Rental}}	 \\ \hline[3pt]
					\SetCell{LightGreen}rentalID & INT	&  	Jedinstveni identifikator najma romobila	\\ \hline
					\SetCell{LightBlue}userID & INT	&  	Jedinstveni identifikator korisnika  \\ \hline
					\SetCell{LightBlue}scooterID & INT	&  	Jedinstveni identifikator romobila  \\ \hline
					status & VARCHAR & Trenutni status najma romobila  \\ \hline
					rentalTimeStart & VARCHAR & Početno vrijeme najma romobila  \\ \hline
					rentalTimeEnd & VARCHAR & Završno vrijeme najma romobila  \\ \hline
				\end{longtblr}
				
				\begin{longtblr}[
					label=none,
					entry=none
					]{
						width = \textwidth,
						colspec={|X[10,l]|X[8, l]|X[20, l]|}, 
						rowhead = 1,
					} %definicija širine tablice, širine stupaca, poravnanje i broja redaka naslova tablice
					\hline \SetCell[c=3]{c}{\textbf{Rating}}	 \\ \hline[3pt]
					\SetCell{LightGreen}ratingID & INT	&  	Jedinstveni identifikator ocjene	\\ \hline
					\SetCell{LightBlue}rentalID & INT	&  	Jedinstveni identifikator najma romobila	\\ \hline
					grade & VARCHAR & Ocjena vožnje romobilom  \\ \hline
					comment & VARCHAR & Komentar nakon vožnje romobilom  \\ \hline
					ratingTime & VARCHAR & Vrijeme ocjenjivanja vožnje \\ \hline
				\end{longtblr}
				
				\begin{longtblr}[
					label=none,
					entry=none
					]{
						width = \textwidth,
						colspec={|X[10,l]|X[8, l]|X[20, l]|}, 
						rowhead = 1,
					} %definicija širine tablice, širine stupaca, poravnanje i broja redaka naslova tablice
					\hline \SetCell[c=3]{c}{\textbf{Transaction}}	 \\ \hline[3pt]
					\SetCell{LightGreen}transactionID & INT	&  	Jedinstveni identifikator transakcije	\\ \hline
					\SetCell{LightBlue}rentalID & INT	&  	Jedinstveni identifikator najma romobila	\\ \hline
					kilometersPassed & VARCHAR & Kilometri prijeđeni romobilom  \\ \hline
					timeOfTransaction & VARCHAR &  Vrijeme transakcije  \\ \hline
					status & VARCHAR & Staus transakcije \\ \hline
				\end{longtblr}
				
				
			
			\subsection{Dijagram baze podataka}
				\textit{ U ovom potpoglavlju potrebno je umetnuti dijagram baze podataka. Primarni i strani ključevi moraju biti označeni, a tablice povezane. Bazu podataka je potrebno normalizirati. Podsjetite se kolegija "Baze podataka".}
			
			\eject
			
			
		\section{Dijagram razreda}
		
			\textit{Potrebno je priložiti dijagram razreda s pripadajućim opisom. Zbog preglednosti je moguće dijagram razlomiti na više njih, ali moraju biti grupirani prema sličnim razinama apstrakcije i srodnim funkcionalnostima.}\\
			
			\textbf{\textit{dio 1. revizije}}\\
			
			\textit{Prilikom prve predaje projekta, potrebno je priložiti potpuno razrađen dijagram razreda vezan uz \textbf{generičku funkcionalnost} sustava. Ostale funkcionalnosti trebaju biti idejno razrađene u dijagramu sa sljedećim komponentama: nazivi razreda, nazivi metoda i vrste pristupa metodama (npr. javni, zaštićeni), nazivi atributa razreda, veze i odnosi između razreda.}\\
			
			\textbf{\textit{dio 2. revizije}}\\			
			
			\textit{Prilikom druge predaje projekta dijagram razreda i opisi moraju odgovarati stvarnom stanju implementacije}
			
			
			
			\eject
		
		\section{Dijagram stanja}
			
			
			\textbf{\textit{dio 2. revizije}}\\
			
			\textit{Potrebno je priložiti dijagram stanja i opisati ga. Dovoljan je jedan dijagram stanja koji prikazuje \textbf{značajan dio funkcionalnosti} sustava. Na primjer, stanja korisničkog sučelja i tijek korištenja neke ključne funkcionalnosti jesu značajan dio sustava, a registracija i prijava nisu. }
			
			
			\eject 
		
		\section{Dijagram aktivnosti}
			
			\textbf{\textit{dio 2. revizije}}\\
			
			 \textit{Potrebno je priložiti dijagram aktivnosti s pripadajućim opisom. Dijagram aktivnosti treba prikazivati značajan dio sustava.}
			
			\eject
		\section{Dijagram komponenti}
		
			\textbf{\textit{dio 2. revizije}}\\
		
			 \textit{Potrebno je priložiti dijagram komponenti s pripadajućim opisom. Dijagram komponenti treba prikazivati strukturu cijele aplikacije.}